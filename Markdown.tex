% Options for packages loaded elsewhere
\PassOptionsToPackage{unicode}{hyperref}
\PassOptionsToPackage{hyphens}{url}
%
\documentclass[
]{article}
\usepackage{amsmath,amssymb}
\usepackage{lmodern}
\usepackage{iftex}
\ifPDFTeX
  \usepackage[T1]{fontenc}
  \usepackage[utf8]{inputenc}
  \usepackage{textcomp} % provide euro and other symbols
\else % if luatex or xetex
  \usepackage{unicode-math}
  \defaultfontfeatures{Scale=MatchLowercase}
  \defaultfontfeatures[\rmfamily]{Ligatures=TeX,Scale=1}
\fi
% Use upquote if available, for straight quotes in verbatim environments
\IfFileExists{upquote.sty}{\usepackage{upquote}}{}
\IfFileExists{microtype.sty}{% use microtype if available
  \usepackage[]{microtype}
  \UseMicrotypeSet[protrusion]{basicmath} % disable protrusion for tt fonts
}{}
\makeatletter
\@ifundefined{KOMAClassName}{% if non-KOMA class
  \IfFileExists{parskip.sty}{%
    \usepackage{parskip}
  }{% else
    \setlength{\parindent}{0pt}
    \setlength{\parskip}{6pt plus 2pt minus 1pt}}
}{% if KOMA class
  \KOMAoptions{parskip=half}}
\makeatother
\usepackage{xcolor}
\usepackage[margin=1in]{geometry}
\usepackage{color}
\usepackage{fancyvrb}
\newcommand{\VerbBar}{|}
\newcommand{\VERB}{\Verb[commandchars=\\\{\}]}
\DefineVerbatimEnvironment{Highlighting}{Verbatim}{commandchars=\\\{\}}
% Add ',fontsize=\small' for more characters per line
\usepackage{framed}
\definecolor{shadecolor}{RGB}{248,248,248}
\newenvironment{Shaded}{\begin{snugshade}}{\end{snugshade}}
\newcommand{\AlertTok}[1]{\textcolor[rgb]{0.94,0.16,0.16}{#1}}
\newcommand{\AnnotationTok}[1]{\textcolor[rgb]{0.56,0.35,0.01}{\textbf{\textit{#1}}}}
\newcommand{\AttributeTok}[1]{\textcolor[rgb]{0.77,0.63,0.00}{#1}}
\newcommand{\BaseNTok}[1]{\textcolor[rgb]{0.00,0.00,0.81}{#1}}
\newcommand{\BuiltInTok}[1]{#1}
\newcommand{\CharTok}[1]{\textcolor[rgb]{0.31,0.60,0.02}{#1}}
\newcommand{\CommentTok}[1]{\textcolor[rgb]{0.56,0.35,0.01}{\textit{#1}}}
\newcommand{\CommentVarTok}[1]{\textcolor[rgb]{0.56,0.35,0.01}{\textbf{\textit{#1}}}}
\newcommand{\ConstantTok}[1]{\textcolor[rgb]{0.00,0.00,0.00}{#1}}
\newcommand{\ControlFlowTok}[1]{\textcolor[rgb]{0.13,0.29,0.53}{\textbf{#1}}}
\newcommand{\DataTypeTok}[1]{\textcolor[rgb]{0.13,0.29,0.53}{#1}}
\newcommand{\DecValTok}[1]{\textcolor[rgb]{0.00,0.00,0.81}{#1}}
\newcommand{\DocumentationTok}[1]{\textcolor[rgb]{0.56,0.35,0.01}{\textbf{\textit{#1}}}}
\newcommand{\ErrorTok}[1]{\textcolor[rgb]{0.64,0.00,0.00}{\textbf{#1}}}
\newcommand{\ExtensionTok}[1]{#1}
\newcommand{\FloatTok}[1]{\textcolor[rgb]{0.00,0.00,0.81}{#1}}
\newcommand{\FunctionTok}[1]{\textcolor[rgb]{0.00,0.00,0.00}{#1}}
\newcommand{\ImportTok}[1]{#1}
\newcommand{\InformationTok}[1]{\textcolor[rgb]{0.56,0.35,0.01}{\textbf{\textit{#1}}}}
\newcommand{\KeywordTok}[1]{\textcolor[rgb]{0.13,0.29,0.53}{\textbf{#1}}}
\newcommand{\NormalTok}[1]{#1}
\newcommand{\OperatorTok}[1]{\textcolor[rgb]{0.81,0.36,0.00}{\textbf{#1}}}
\newcommand{\OtherTok}[1]{\textcolor[rgb]{0.56,0.35,0.01}{#1}}
\newcommand{\PreprocessorTok}[1]{\textcolor[rgb]{0.56,0.35,0.01}{\textit{#1}}}
\newcommand{\RegionMarkerTok}[1]{#1}
\newcommand{\SpecialCharTok}[1]{\textcolor[rgb]{0.00,0.00,0.00}{#1}}
\newcommand{\SpecialStringTok}[1]{\textcolor[rgb]{0.31,0.60,0.02}{#1}}
\newcommand{\StringTok}[1]{\textcolor[rgb]{0.31,0.60,0.02}{#1}}
\newcommand{\VariableTok}[1]{\textcolor[rgb]{0.00,0.00,0.00}{#1}}
\newcommand{\VerbatimStringTok}[1]{\textcolor[rgb]{0.31,0.60,0.02}{#1}}
\newcommand{\WarningTok}[1]{\textcolor[rgb]{0.56,0.35,0.01}{\textbf{\textit{#1}}}}
\usepackage{graphicx}
\makeatletter
\def\maxwidth{\ifdim\Gin@nat@width>\linewidth\linewidth\else\Gin@nat@width\fi}
\def\maxheight{\ifdim\Gin@nat@height>\textheight\textheight\else\Gin@nat@height\fi}
\makeatother
% Scale images if necessary, so that they will not overflow the page
% margins by default, and it is still possible to overwrite the defaults
% using explicit options in \includegraphics[width, height, ...]{}
\setkeys{Gin}{width=\maxwidth,height=\maxheight,keepaspectratio}
% Set default figure placement to htbp
\makeatletter
\def\fps@figure{htbp}
\makeatother
\setlength{\emergencystretch}{3em} % prevent overfull lines
\providecommand{\tightlist}{%
  \setlength{\itemsep}{0pt}\setlength{\parskip}{0pt}}
\setcounter{secnumdepth}{-\maxdimen} % remove section numbering
\ifLuaTeX
  \usepackage{selnolig}  % disable illegal ligatures
\fi
\IfFileExists{bookmark.sty}{\usepackage{bookmark}}{\usepackage{hyperref}}
\IfFileExists{xurl.sty}{\usepackage{xurl}}{} % add URL line breaks if available
\urlstyle{same} % disable monospaced font for URLs
\hypersetup{
  pdftitle={UFRPE - Estatistica 1 VA},
  pdfauthor={Augusto Miranda - UFRPE},
  hidelinks,
  pdfcreator={LaTeX via pandoc}}

\title{UFRPE - Estatistica 1 VA}
\author{Augusto Miranda - UFRPE}
\date{2023-02-09}

\begin{document}
\maketitle

Carregar Pacotes

\begin{Shaded}
\begin{Highlighting}[]
\FunctionTok{library}\NormalTok{(readxl)}
\FunctionTok{library}\NormalTok{(dplyr)}
\end{Highlighting}
\end{Shaded}

\begin{verbatim}
## 
## Attaching package: 'dplyr'
\end{verbatim}

\begin{verbatim}
## The following objects are masked from 'package:stats':
## 
##     filter, lag
\end{verbatim}

\begin{verbatim}
## The following objects are masked from 'package:base':
## 
##     intersect, setdiff, setequal, union
\end{verbatim}

\begin{Shaded}
\begin{Highlighting}[]
\FunctionTok{library}\NormalTok{(tidyverse)}
\end{Highlighting}
\end{Shaded}

\begin{verbatim}
## -- Attaching packages --------------------------------------- tidyverse 1.3.2
## --
\end{verbatim}

\begin{verbatim}
## v ggplot2 3.4.0     v purrr   1.0.1
## v tibble  3.1.8     v stringr 1.5.0
## v tidyr   1.3.0     v forcats 1.0.0
## v readr   2.1.3     
## -- Conflicts ------------------------------------------ tidyverse_conflicts() --
## x dplyr::filter() masks stats::filter()
## x dplyr::lag()    masks stats::lag()
\end{verbatim}

\begin{Shaded}
\begin{Highlighting}[]
\FunctionTok{library}\NormalTok{(ggplot2)}
\end{Highlighting}
\end{Shaded}

\hypertarget{r-markdown}{%
\subsection{R Markdown}\label{r-markdown}}

This is an R Markdown document. Markdown is a simple formatting syntax
for authoring HTML, PDF, and MS Word documents. For more details on
using R Markdown see \url{http://rmarkdown.rstudio.com}.

When you click the \textbf{Knit} button a document will be generated
that includes both content as well as the output of any embedded R code
chunks within the document. You can embed an R code chunk like this:

\begin{Shaded}
\begin{Highlighting}[]
\CommentTok{\# LENDO ARQUIVOS XLS/XLSX {-} EXCEL}

\CommentTok{\#dadosPraia = read\_excel(file.choose())}
\NormalTok{dadosPraia }\OtherTok{=} \FunctionTok{read\_excel}\NormalTok{(}\StringTok{"sp\_beaches.xlsx"}\NormalTok{)}
\NormalTok{dadosPraia}
\end{Highlighting}
\end{Shaded}

\begin{verbatim}
## # A tibble: 74,056 x 4
##    City     Beach                   Date       Enterococcus
##    <chr>    <chr>                   <chr>      <chr>       
##  1 BERTIOGA BORACÉIA - COL. MARISTA 2012-01-03 8.0         
##  2 BERTIOGA BORACÉIA - COL. MARISTA 2012-01-08 22.0        
##  3 BERTIOGA BORACÉIA - COL. MARISTA 2012-01-15 17.0        
##  4 BERTIOGA BORACÉIA - COL. MARISTA 2012-01-22 8.0         
##  5 BERTIOGA BORACÉIA - COL. MARISTA 2012-01-29 2.0         
##  6 BERTIOGA BORACÉIA - COL. MARISTA 2012-02-05 1.0         
##  7 BERTIOGA BORACÉIA - COL. MARISTA 2012-02-12 68.0        
##  8 BERTIOGA BORACÉIA - COL. MARISTA 2012-02-19 32.0        
##  9 BERTIOGA BORACÉIA - COL. MARISTA 2012-02-26 1.0         
## 10 BERTIOGA BORACÉIA - COL. MARISTA 2012-03-04 1.0         
## # ... with 74,046 more rows
\end{verbatim}

\begin{Shaded}
\begin{Highlighting}[]
\FunctionTok{View}\NormalTok{(dadosPraia)}




\CommentTok{\#VERIFICANDO NOMES DAS TABELAS}
\FunctionTok{names}\NormalTok{(dadosPraia)}
\end{Highlighting}
\end{Shaded}

\begin{verbatim}
## [1] "City"         "Beach"        "Date"         "Enterococcus"
\end{verbatim}

\begin{Shaded}
\begin{Highlighting}[]
\CommentTok{\#FILTAR DADOS {-} AUGUSTO MIRANDA {-} PRAIA PERUÍBE}

\NormalTok{Cidade }\OtherTok{\textless{}{-}}\FunctionTok{filter}\NormalTok{(dadosPraia, City }\SpecialCharTok{==} \StringTok{"PERUÍBE"}\NormalTok{)}


\DocumentationTok{\#\#\#\#\#\#\#\#\#\#\#\#\#\#\#\#\#\#\#\#\#\#\#\#\#\#\#\#\#\#\#\#\#\#\#\#\#\#\#\#\#\#\#\#\#\#\#\#\#\#\#\#\#\#\#\#\#\#\#\#\#\#\#\#\#\#\#\#\#\#\#\#\#\#\#\#\#\#\#}
\DocumentationTok{\#\#\#\#\#\# CONVERTENDO ULTIMA COLUNA PARA NUMERIC}

\NormalTok{Cidade}\SpecialCharTok{$}\NormalTok{Enterococcus  }\OtherTok{\textless{}{-}} \FunctionTok{as.numeric}\NormalTok{(}\FunctionTok{as.character}\NormalTok{(Cidade}\SpecialCharTok{$}\NormalTok{Enterococcus))}
\FunctionTok{str}\NormalTok{(Cidade)}
\end{Highlighting}
\end{Shaded}

\begin{verbatim}
## tibble [2,725 x 4] (S3: tbl_df/tbl/data.frame)
##  $ City        : chr [1:2725] "PERUÍBE" "PERUÍBE" "PERUÍBE" "PERUÍBE" ...
##  $ Beach       : chr [1:2725] "PERUÍBE (R. ICARAÍBA)" "PERUÍBE (R. ICARAÍBA)" "PERUÍBE (R. ICARAÍBA)" "PERUÍBE (R. ICARAÍBA)" ...
##  $ Date        : chr [1:2725] "2012-01-03" "2012-01-08" "2012-01-15" "2012-01-22" ...
##  $ Enterococcus: num [1:2725] 5 13 1 36 10 5 272 32 1 10 ...
\end{verbatim}

\begin{Shaded}
\begin{Highlighting}[]
\FunctionTok{View}\NormalTok{(Cidade)}
\FunctionTok{sapply}\NormalTok{(Cidade, class)}
\end{Highlighting}
\end{Shaded}

\begin{verbatim}
##         City        Beach         Date Enterococcus 
##  "character"  "character"  "character"    "numeric"
\end{verbatim}

\begin{Shaded}
\begin{Highlighting}[]
\FunctionTok{str}\NormalTok{(Cidade)}
\end{Highlighting}
\end{Shaded}

\begin{verbatim}
## tibble [2,725 x 4] (S3: tbl_df/tbl/data.frame)
##  $ City        : chr [1:2725] "PERUÍBE" "PERUÍBE" "PERUÍBE" "PERUÍBE" ...
##  $ Beach       : chr [1:2725] "PERUÍBE (R. ICARAÍBA)" "PERUÍBE (R. ICARAÍBA)" "PERUÍBE (R. ICARAÍBA)" "PERUÍBE (R. ICARAÍBA)" ...
##  $ Date        : chr [1:2725] "2012-01-03" "2012-01-08" "2012-01-15" "2012-01-22" ...
##  $ Enterococcus: num [1:2725] 5 13 1 36 10 5 272 32 1 10 ...
\end{verbatim}

\begin{Shaded}
\begin{Highlighting}[]
\DocumentationTok{\#\#\#\#\#\#\#\#\#\#\#\#\#\#\#\#\#\#\#\#\#\#\#\#\#\#\#\#\#\#\#\#\#\#\#\#\#\#\#\#\#\#\#\#\#\#\#\#\#\#\#\#\#\#\#\#\#\#\#\#\#\#\#\#\#\#\#\#\#\#\#\#\#\#\#\#\# dadosPraia \%\textgreater{}\% reframe(dadosPraia, media = mean(Enterococcus))}
\CommentTok{\# Questão 1:}
\CommentTok{\#   Encontre média, desvio{-}padrão, mediana, Q1, Q3, }
\CommentTok{\#   mínimo e máximo dos enterococos de cada praia (summarise)}
\CommentTok{\#}
\CommentTok{\#}
\CommentTok{\#}
\CommentTok{\# }
\DocumentationTok{\#\# RESPOSTA: }
\CommentTok{\# MEDIA, MEDIANA, DESVIO PADRÃO}


\CommentTok{\#MÉDIA}
\NormalTok{Cidade }\SpecialCharTok{\%\textgreater{}\%} \FunctionTok{group\_by}\NormalTok{(Beach) }\SpecialCharTok{\%\textgreater{}\%} \FunctionTok{summarise}\NormalTok{(}\AttributeTok{Media =} \FunctionTok{mean}\NormalTok{(Enterococcus))}
\end{Highlighting}
\end{Shaded}

\begin{verbatim}
## # A tibble: 6 x 2
##   Beach                            Media
##   <chr>                            <dbl>
## 1 GUARAÚ                            42.9
## 2 PERUÍBE (AV. S. JOÃO)             62.5
## 3 PERUÍBE (BALN. SÃO JOÃO BATISTA)  60.5
## 4 PERUÍBE (PARQUE TURÍSTICO)        54.5
## 5 PERUÍBE (R. ICARAÍBA)             50.8
## 6 PRAINHA                           52.6
\end{verbatim}

\begin{Shaded}
\begin{Highlighting}[]
\CommentTok{\#MEDIANA}
\NormalTok{Cidade }\SpecialCharTok{\%\textgreater{}\%} \FunctionTok{group\_by}\NormalTok{(Beach) }\SpecialCharTok{\%\textgreater{}\%} \FunctionTok{summarise}\NormalTok{(}\AttributeTok{Mediana =} \FunctionTok{median}\NormalTok{(Enterococcus))}
\end{Highlighting}
\end{Shaded}

\begin{verbatim}
## # A tibble: 6 x 2
##   Beach                            Mediana
##   <chr>                              <dbl>
## 1 GUARAÚ                                 7
## 2 PERUÍBE (AV. S. JOÃO)                 19
## 3 PERUÍBE (BALN. SÃO JOÃO BATISTA)      14
## 4 PERUÍBE (PARQUE TURÍSTICO)            12
## 5 PERUÍBE (R. ICARAÍBA)                 12
## 6 PRAINHA                               12
\end{verbatim}

\begin{Shaded}
\begin{Highlighting}[]
\DocumentationTok{\#\# DESVIO PADRÃO}
\NormalTok{Cidade }\SpecialCharTok{\%\textgreater{}\%} \FunctionTok{group\_by}\NormalTok{(Beach) }\SpecialCharTok{\%\textgreater{}\%} \FunctionTok{summarise}\NormalTok{(}\AttributeTok{DesvioP =}\NormalTok{ (Enterococcus }\SpecialCharTok{{-}} \FunctionTok{mean}\NormalTok{(Enterococcus)))}
\end{Highlighting}
\end{Shaded}

\begin{verbatim}
## Warning: Returning more (or less) than 1 row per `summarise()` group was deprecated in
## dplyr 1.1.0.
## i Please use `reframe()` instead.
## i When switching from `summarise()` to `reframe()`, remember that `reframe()`
##   always returns an ungrouped data frame and adjust accordingly.
\end{verbatim}

\begin{verbatim}
## `summarise()` has grouped output by 'Beach'. You can override using the
## `.groups` argument.
\end{verbatim}

\begin{verbatim}
## # A tibble: 2,725 x 2
## # Groups:   Beach [6]
##    Beach  DesvioP
##    <chr>    <dbl>
##  1 GUARAÚ  -41.9 
##  2 GUARAÚ  -33.9 
##  3 GUARAÚ   -2.85
##  4 GUARAÚ  -18.9 
##  5 GUARAÚ  -39.9 
##  6 GUARAÚ  -39.9 
##  7 GUARAÚ  -25.9 
##  8 GUARAÚ  -18.9 
##  9 GUARAÚ  -35.9 
## 10 GUARAÚ  -15.9 
## # ... with 2,715 more rows
\end{verbatim}

\begin{Shaded}
\begin{Highlighting}[]
\CommentTok{\#MÉDIA E MEDIANA JUNTAS}
\NormalTok{Cidade }\SpecialCharTok{\%\textgreater{}\%} \FunctionTok{group\_by}\NormalTok{(Beach) }\SpecialCharTok{\%\textgreater{}\%} \FunctionTok{summarise}\NormalTok{(}\AttributeTok{Media =} \FunctionTok{mean}\NormalTok{(Enterococcus), }\AttributeTok{Mediana =} \FunctionTok{median}\NormalTok{(Enterococcus))}
\end{Highlighting}
\end{Shaded}

\begin{verbatim}
## # A tibble: 6 x 3
##   Beach                            Media Mediana
##   <chr>                            <dbl>   <dbl>
## 1 GUARAÚ                            42.9       7
## 2 PERUÍBE (AV. S. JOÃO)             62.5      19
## 3 PERUÍBE (BALN. SÃO JOÃO BATISTA)  60.5      14
## 4 PERUÍBE (PARQUE TURÍSTICO)        54.5      12
## 5 PERUÍBE (R. ICARAÍBA)             50.8      12
## 6 PRAINHA                           52.6      12
\end{verbatim}

\begin{Shaded}
\begin{Highlighting}[]
\CommentTok{\#Quartil {-} }
\FunctionTok{quantile}\NormalTok{(Cidade}\SpecialCharTok{$}\NormalTok{Enterococcus, }\AttributeTok{probs =} \FunctionTok{seq}\NormalTok{(}\DecValTok{0}\NormalTok{,}\DecValTok{1}\NormalTok{, }\FloatTok{0.25}\NormalTok{))}
\end{Highlighting}
\end{Shaded}

\begin{verbatim}
##   0%  25%  50%  75% 100% 
##    1    4   12   57 1180
\end{verbatim}

\begin{Shaded}
\begin{Highlighting}[]
\NormalTok{Cidade }\SpecialCharTok{\%\textgreater{}\%} \FunctionTok{summarise}\NormalTok{(}\FunctionTok{quantile}\NormalTok{(Enterococcus, }\AttributeTok{probs =} \FunctionTok{seq}\NormalTok{(}\DecValTok{0}\NormalTok{,}\DecValTok{1}\NormalTok{, }\FloatTok{0.25}\NormalTok{)))}
\end{Highlighting}
\end{Shaded}

\begin{verbatim}
## Warning: Returning more (or less) than 1 row per `summarise()` group was deprecated in
## dplyr 1.1.0.
## i Please use `reframe()` instead.
## i When switching from `summarise()` to `reframe()`, remember that `reframe()`
##   always returns an ungrouped data frame and adjust accordingly.
\end{verbatim}

\begin{verbatim}
## # A tibble: 5 x 1
##   `quantile(Enterococcus, probs = seq(0, 1, 0.25))`
##                                               <dbl>
## 1                                                 1
## 2                                                 4
## 3                                                12
## 4                                                57
## 5                                              1180
\end{verbatim}

\hypertarget{including-plots}{%
\subsection{Including Plots}\label{including-plots}}

\begin{Shaded}
\begin{Highlighting}[]
\DocumentationTok{\#\#\#\#\#\#\#\#\#\#\#\#\#\#\#\#\#\#\#\#\#\#\#\#\#\#\#\#\#\#\#\#\#\#\#\#\#\#\#\#\#\#\#\#\#\#\#\#\#\#\#\#\#\#\#\#\#\#\#\#\#\#\#\#\#\#\#\#\#\#\#\#\#\#\#\#\# }
\CommentTok{\# Questão 2:}
\CommentTok{\#   Faça um gráfico de barras com a variável Beach, ordenando da praia }
\CommentTok{\#   com maior quantidades de amostras para a menor. Colorir gráfico com }
\CommentTok{\#   base na praia. Anotar as porcentagens no topo das barras (ggplot2).}
\CommentTok{\#}
\CommentTok{\# install.packages("ggplot2")}

\NormalTok{Cidade }\SpecialCharTok{\%\textgreater{}\%} \FunctionTok{group\_by}\NormalTok{(Beach) }\SpecialCharTok{\%\textgreater{}\%} \FunctionTok{summarise}\NormalTok{(}\AttributeTok{Contagem =} \FunctionTok{n}\NormalTok{()) }\SpecialCharTok{\%\textgreater{}\%} \FunctionTok{ggplot}\NormalTok{(}\FunctionTok{aes}\NormalTok{(}\AttributeTok{x =}\NormalTok{ Beach, }\AttributeTok{y =}\NormalTok{ Contagem, }\AttributeTok{color =}\NormalTok{ Beach, }\AttributeTok{label =}\NormalTok{ Contagem, }\AttributeTok{fill =}\NormalTok{ Beach)) }\SpecialCharTok{+} \FunctionTok{geom\_bar}\NormalTok{(}\AttributeTok{stat =} \StringTok{"Identity"}\NormalTok{) }\SpecialCharTok{+} \FunctionTok{geom\_label}\NormalTok{() }\SpecialCharTok{+} \FunctionTok{coord\_flip}\NormalTok{() }
\end{Highlighting}
\end{Shaded}

\includegraphics{Markdown_files/figure-latex/unnamed-chunk-3-1.pdf}

\begin{Shaded}
\begin{Highlighting}[]
\DocumentationTok{\#\#\#\#\#\#\#\#\#\#\#\#\#\#\#\#\#\#\#\#\#\#\#\#\#\#\#\#\#\#\#\#\#\#\#\#\#\#\#\#\#\#\#\#\#\#\#\#\#\#\#\#\#\#\#\#\#\#\#\#\#\#\#\#\#\#\#\#\#\#\#\#\#\#\#\#\# }
\CommentTok{\# Questão 3:}
\CommentTok{\#   Repita a questão 2, fazendo desta vez um gráfico de pizza (ggplot2).}
\CommentTok{\#}
\CommentTok{\#}
\NormalTok{Cidade }\SpecialCharTok{\%\textgreater{}\%} \FunctionTok{group\_by}\NormalTok{(Beach) }\SpecialCharTok{\%\textgreater{}\%} \FunctionTok{summarise}\NormalTok{(}\AttributeTok{Contagem =} \FunctionTok{n}\NormalTok{()) }\SpecialCharTok{\%\textgreater{}\%} \FunctionTok{ggplot}\NormalTok{(}\FunctionTok{aes}\NormalTok{(}\AttributeTok{x =}\NormalTok{ Beach, }\AttributeTok{y =}\NormalTok{ Contagem, }\AttributeTok{color =}\NormalTok{ Beach, }\AttributeTok{label =}\NormalTok{ Contagem, }\AttributeTok{fill =}\NormalTok{ Beach)) }\SpecialCharTok{+} \FunctionTok{geom\_bar}\NormalTok{(}\AttributeTok{stat =} \StringTok{"Identity"}\NormalTok{) }\SpecialCharTok{+} \FunctionTok{geom\_label}\NormalTok{() }\SpecialCharTok{+} \FunctionTok{coord\_polar}\NormalTok{(}\StringTok{"y"}\NormalTok{, }\AttributeTok{start =} \DecValTok{0}\NormalTok{) }
\end{Highlighting}
\end{Shaded}

\includegraphics{Markdown_files/figure-latex/unnamed-chunk-3-2.pdf}

\begin{Shaded}
\begin{Highlighting}[]
\DocumentationTok{\#\#\#\#\#\#\#\#\#\#\#\#\#\#\#\#\#\#\#\#\#\#\#\#\#\#\#\#\#\#\#\#\#\#\#\#\#\#\#\#\#\#\#\#\#\#\#\#\#\#\#\#\#\#\#\#\#\#\#\#\#\#\#\#\#\#\#\#\#\#\#\#\#\#\#\#\# }
\CommentTok{\# Questão 4:  }
\CommentTok{\# Fazer um histograma com todos os dados de enterococos das praias da sua cidade (ggplot2).}
\CommentTok{\#}
\NormalTok{Cidade }\SpecialCharTok{\%\textgreater{}\%} \FunctionTok{group\_by}\NormalTok{(Beach) }\SpecialCharTok{\%\textgreater{}\%} \FunctionTok{summarise}\NormalTok{(}\AttributeTok{Contagem =} \FunctionTok{n}\NormalTok{()) }\SpecialCharTok{\%\textgreater{}\%} \FunctionTok{ggplot}\NormalTok{(}\FunctionTok{aes}\NormalTok{(}\AttributeTok{x =}\NormalTok{ Beach, }\AttributeTok{y =}\NormalTok{ Contagem, }\AttributeTok{color =}\NormalTok{ Beach, }\AttributeTok{label =}\NormalTok{ Contagem, }\AttributeTok{fill =}\NormalTok{ Beach)) }\SpecialCharTok{+} \FunctionTok{geom\_histogram}\NormalTok{(}\AttributeTok{stat =} \StringTok{"Identity"}\NormalTok{)}
\end{Highlighting}
\end{Shaded}

\begin{verbatim}
## Warning in geom_histogram(stat = "Identity"): Ignoring unknown parameters:
## `binwidth`, `bins`, and `pad`
\end{verbatim}

\includegraphics{Markdown_files/figure-latex/unnamed-chunk-3-3.pdf}

\begin{Shaded}
\begin{Highlighting}[]
\DocumentationTok{\#\#\#\#\#\#\#\#\#\#\#\#\#\#\#\#\#\#\#\#\#\#\#\#\#\#\#\#\#\#\#\#\#\#\#\#\#\#\#\#\#\#\#\#\#\#\#\#\#\#\#\#\#\#\#\#\#\#\#\#\#\#\#\#\#\#\#\#\#\#\#\#\#\#\#\#\# }
\CommentTok{\# Questão 5:}
\CommentTok{\#   Fazer box{-}plots de todas as praias da sua cidade num único gráfico (ggplot2).}
\CommentTok{\#}


\NormalTok{Cidade }\SpecialCharTok{\%\textgreater{}\%} \FunctionTok{ggplot}\NormalTok{(}\FunctionTok{aes}\NormalTok{(}\FunctionTok{reorder}\NormalTok{(Beach, Enterococcus), Enterococcus)) }\SpecialCharTok{+} \FunctionTok{geom\_boxplot}\NormalTok{(}\AttributeTok{fill =} \StringTok{"red"}\NormalTok{, }\AttributeTok{alpha =} \FloatTok{0.5}\NormalTok{)}
\end{Highlighting}
\end{Shaded}

\includegraphics{Markdown_files/figure-latex/unnamed-chunk-3-4.pdf}

\end{document}
